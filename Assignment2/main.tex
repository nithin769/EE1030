%iffalse
\let\negmedspace\undefined
\let\negthickspace\undefined
\documentclass[journal,12pt,twocolumn]{IEEEtran}
\usepackage{cite}
\usepackage{amsmath,amssymb,amsfonts,amsthm}
\usepackage{algorithmic}
\usepackage{graphicx}
\usepackage{textcomp}
\usepackage{xcolor}
\usepackage{txfonts}
\usepackage{listings}
\usepackage{enumitem}
\usepackage{mathtools}
\usepackage{gensymb}
\usepackage{comment}
\usepackage[breaklinks=true]{hyperref}
\usepackage{tkz-euclide} 
\usepackage{listings}
\usepackage{gvv}   
%\def\inputGnumericTable{}                                 
\usepackage[latin1]{inputenc}                                
\usepackage{color}                                            
\usepackage{array}                                            
\usepackage{longtable}                                       
\usepackage{calc}                                             
\usepackage{multirow}                                         
\usepackage{hhline}                                           
\usepackage{ifthen}                                           
\usepackage{lscape}
\usepackage{tabularx}
\usepackage{array}
\usepackage{float}


\newtheorem{theorem}{Theorem}[section]
\newtheorem{problem}{Problem}
\newtheorem{proposition}{Proposition}[section]
\newtheorem{lemma}{Lemma}[section]
\newtheorem{corollary}[theorem]{Corollary}
\newtheorem{example}{Example}[section]
\newtheorem{definition}[problem]{Definition}
\newcommand{\BEQA}{\begin{eqnarray}}
\newcommand{\EEQA}{\end{eqnarray}}
\newcommand{\define}{\stackrel{\triangle}{=}}
\theoremstyle{remark}
\newtheorem{rem}{Remark}

% Marks the beginning of the document
\begin{document}
\bibliographystyle{IEEEtran}

\title{Assignment-2 \\ CHAPETR-15 \\ Matrices and Determinants}
\author{EE24BTECH11048-NITHIN.K} 
\maketitle
\newpage
\bigskip

\renewcommand{\thefigure}{\theenumi}
\renewcommand{\thetable}{\theenumi}
\begin{enumerate}
\section{Section-B}
%1
	\item Let A= $\myvec{
			0 & 0 & -1 \\
			0 & -1 & 0 \\
			-1 & 0 & 0}$. The only correct statement about the matrix A is \hfill{[2004]}
		\begin{enumerate}
			\item $A^2=I$
			\item $A = \brak{-1}I$,where I is a unit matrix
			\item $A^{-1}$ does not exist
			\item A is a zero matrix \\
		\end{enumerate}
%2		
	\item Let A= $\myvec{
			1 & -1 & 1 \\
			2 & 1 & -3 \\
			1 & 1 & 1}$, and 10B= $\myvec{
			4 & 2 & 2 \\
			-5 & 0 & \alpha \\
			1 & -2 & 3}$. If B is the inversion of matrix A, then $\alpha$ is \hfill{[2004]}
		\begin{enumerate}
			\item 5
			\item -1
			\item 2
			\item -2 \\
		\end{enumerate}
%3
	\item If $a_1,a_2,a_3,\dots,a_n,\dots$ are in G.P, then the value of the determinant \hfill{[2004]}
		$\mydet{
			loga_{n} & loga_{n+1} & loga_{n+2} \\
			loga_{n+3} & loga_{n+4} & loga_{n+5} \\
			loga_{n+6} & loga_{n+7} & loga_{n+8}}$, is
		\begin{enumerate}
			\item -2
			\item 1
			\item 2
			\item 0 \\
		\end{enumerate}
%4		
	\item If $A^2-A+I=0$, then the inverse of A is \hfill{[2005]}
		\begin{enumerate}
			\item A+I
			\item A
			\item A-I
			\item I-A \\
		\end{enumerate}
%5
	\item The system of equations \\
		$\alpha x+y+z = \alpha -1$ \\
		$x+ \alpha y+z = \alpha -1$ \\
		$x+y+ \alpha z = \alpha -1$ \\
		has infinite solutions, if $\alpha$ is \hfill{[2005]}
		\begin{enumerate}
			\item -2
			\item either -2 or 1
			\item not -2
			\item 1 \\
		\end{enumerate}
%6		
	\item If $a^2+b^2+c^2=-2$ and \hfill{[2005]} \\
		f\brak{x}= $\mydet{
			1+a^2x & \brak{1+b^2}x & \brak{1+c^2}x \\
			\brak{1+a^2}x & 1+b^2x & \brak{1+c^2}x \\
			\brak{1+a^2}x & \brak{1+b^2}x & 1+c^2x}$ then $f\brak{x}$ is a polynomial of degree
		\begin{enumerate}
			\item 1
			\item 0
			\item 3
			\item 2 \\
		\end{enumerate}
%7
	\item If $a_1,a_2,a_3,\dots,a_n,\dots$ are in G.P,then the determinant \hfill{[2005]} \\
		$\Delta$= $\mydet{
			loga_n & loga_{n+1} & loga_{n+2} \\
			loga_{n+3} & loga_{n+4} & loga_{n+5} \\
			loga_{n+6} & loga_{n+7} & loga_{n+8}}$ is equal to \\
		\begin{enumerate}
			\item 1
			\item 0
			\item 4
			\item 2 \\
		\end{enumerate}
%8
	\item If A and B are square matrices of size n x n such that $A^2-B^2=\brak{A-B}\brak{A+B}$, then which of the following will be always true? \hfill{[2006]}
		\begin{enumerate}
			\item $A=B$
			\item $AB=BA$
			\item either of A or B is zero matrix
			\item either of A or B is identity matrix \\
		\end{enumerate}
%9
	\item Let A= $\myvec{
			1 & 2 \\
			3 & 4}$ and B= $\myvec{
			a & 0 \\
			0 & b}$, a,b $\in$ N. Then \hfill{[2006]}
		\begin{enumerate}
			\item there cannot exist any B such that $AB=BA$
			\item there exist more than one but finite number of B's such that $AB=BA$
			\item there exists exactly one B such that $AB=BA$
			\item there exist infinitely many B's such that $AB=BA$ \\
		\end{enumerate}
%10
	\item If D= $\mydet{
			1 & 1 & 1 \\
			1 & 1+x & 1 \\
			1 & 1 & 1+y}$ for $x \neq 0$, $y \neq 0$, then D is \hfill{[2007]}
		\begin{enumerate}
			\item divisible by x but not y
			\item divisible by y but not x
			\item divisible neither by x nor y
			\item divisible by both x and y \\
		\end{enumerate}
%11
	\item Let A= $\mydet{
			5 & 5\alpha & \alpha \\
			0 & \alpha & 5\alpha \\
			0 & 0 & 5}$. If $\abs{A^2} = 25$,then $\abs{\alpha}$ equals \hfill{[2007]}
		\begin{enumerate}
			\item $\frac{1}{5}$
			\item 5
			\item $5^2$
			\item 1 \\
		\end{enumerate}
%12
	\item Let A be a 2x2 matrix with real entries. Let I be the 2x2 identity matrix. Denote by tr\brak{A},the sum of diagonal entries of A. Assume that $A^2 = I$. \hfill{[2008]} \\
		Statement-1 : If $A \neq I$ and $A \neq -I$,then det\brak{A}=-1 \\
		Statement-2 : If $A \neq I$ and $A \neq -I$,then $tr\brak{A} \neq 0$.\\
		\begin{enumerate}
			\item Statement-1 is false, Statement-2 is true
			\item Statement-1 is true, Statement-2 is true;Statement-2 is a correct explanation for Statement-1
			\item Statement-1 is true, Statement-2 is true;Statement-2 is not a correct explanation for Statement-1
			\item Statement-1 is true, Statement-2 is false \\
		\end{enumerate}
%13
	\item Let a,b,c be any real numbers. Suppose that there are real numbers x,y,z not all zero such that $x=cy+bz$, $y=az+cx$, and $z=bx+ay$. Then $a^2+b^2+c^2+2abc$ is equal to \hfill{[2008]}
		\begin{enumerate}
			\item 2
			\item -1
			\item 0
			\item 1 \\
		\end{enumerate}
%14
	\item Let A be a square matrix all of whose entries are integers. Then which of the following is true? \hfill{[2008]}
		\begin{enumerate}
			\item If $det\brak{A}\neq\pm1$, then $A^{-1}$ exists but all its entries are not necessarily integers
			\item If $det\brak{A}\neq\pm1$, then $A^{-1}$ exists and all its entries are non integers
			\item If $det\brak{A}=\pm1$, then $A^{-1}$ exists but all its entries are integers
			\item If $det\brak{A}=\pm1$, then $A^{-1}$ need not exists \\
		\end{enumerate}
%15
	\item Let A be a 2x2 matrix \\
		Statement-1:$adj\brak{adjA} = A$ \\
		Statement-2:$\abs{adjA} = \abs{A}$ \hfill{[2009]}
		\begin{enumerate}
			\item Statement-1 is true, Statement-2 is true;Statement-2 is not a correct explanation for Statement-1
			\item Statement-1 is true, Statement-2 is false
			\item Statement-1 is false, Statement-2 is true
			\item Statement-1 is true, Statement-2 is true;Statement-2 is a correct explanation for Statement-1
		\end{enumerate}

\end{enumerate}

\end{document}
