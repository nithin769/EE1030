\let\negmedspace\undefined
\let\negthickspace\undefined
\documentclass[journal]{IEEEtran}
\usepackage[a5paper, margin=10mm, onecolumn]{geometry}
%\usepackage{lmodern} % Ensure lmodern is loaded for pdflatex
\usepackage{tfrupee} % Include tfrupee package

\setlength{\headheight}{1cm} % Set the height of the header box
\setlength{\headsep}{0mm}     % Set the distance between the header box and the top of the text

\usepackage{gvv-book}
\usepackage{gvv}
\usepackage{cite}
\usepackage{amsmath,amssymb,amsfonts,amsthm}
\usepackage{algorithmic}
\usepackage{graphicx}
\usepackage{textcomp}
\usepackage{xcolor}
\usepackage{txfonts}
\usepackage{listings}
\usepackage{enumitem}
\usepackage{mathtools}
\usepackage{gensymb}
\usepackage{comment}
\usepackage[breaklinks=true]{hyperref}
\usepackage{tkz-euclide} 
\usepackage{listings}
% \usepackage{gvv}                                        
\def\inputGnumericTable{}                                 
\usepackage[latin1]{inputenc}                                
\usepackage{color}                                            
\usepackage{array}                                            
\usepackage{longtable}                                       
\usepackage{calc}                                             
\usepackage{multirow}                                         
\usepackage{hhline}                                           
\usepackage{ifthen}                                           
\usepackage{lscape}
\renewcommand{\thefigure}{\theenumi}
\renewcommand{\thetable}{\theenumi}
\setlength{\intextsep}{10pt} % Space between text and floats


\numberwithin{equation}{enumi}
\numberwithin{figure}{enumi}
\renewcommand{\thetable}{\theenumi}

% Marks the beginning of the document
\begin{document}
\bibliographystyle{IEEEtran}

\title{Assignment-5}
\author{EE24BTECH11048-NITHIN.K}
% \maketitle
% \newpage
% \bigskip
{\let\newpage\relax\maketitle}

\begin{enumerate}
\section{SECTION-A}
%16
\item Let the line passing through the point P$\brak{2,-1,2}$ and Q$\brak{5,3,4}$ meet the plane $x-y+z=4$ at the point T. Then the distance of the point R from the plane $x+2y+3z+2=0$ measured parallel
to the line $\frac{x-7}{2} = \frac{y+3}{2} = \frac{z-2}{1}$ is equal to
\begin{enumerate}
\item 3
\item $\sqrt{61}$
\item $\sqrt{31}$
\item $\sqrt{189}$
\end{enumerate}

%17
\item Let the function f : $\sbrak{0,2} \rightarrow$ R be defined as \\
	$f\brak{x} = \begin{cases} e^{min\brak{x^2,x-\sbrak{x}}} & \text{, } x \in \lsbrak{0},\rbrak{1} \\ e^{\sbrak{x-\log_e{x}}} & \text{,} x \in \lsbrak{1},\rbrak{2} \end{cases}$ \\
		where $\sbrak{t}$ denotes the greatest integer less than or equal to t. Then the value of the integral $\int_{0}^{2}xf\brak{x}dx$ is
\begin{enumerate}
\item $\brak{e-1}\brak{e^2+\frac{1}{2}}$
\item $1+\frac{3e}{2}$
\item $2e-\frac{1}{2}$
\item 2e - 1
\end{enumerate}

%18
\item For a$\in$C, let A = $\cbrak{z\in C:Re\brak{a+\vec{z}} > Im\brak{\vec{a}+z}}$ and B = $\cbrak{z\in C:Re\brak{a+\vec{z}} < Im\brak{\vec{a}+z}}$. Then among the two statements: \\
$\brak{S_1} : If Re\brak{a}, Im\brak{a} > 0$, then the set A contains all the real numbers \\
$\brak{S_2} : If Re\brak{a}, Im\brak{a} < 0$, then the set B contains all the real numbers
\begin{enumerate}
\item only $S_1$ is true		
\item both are false
\item only $S_2$ is true
\item both are true
\end{enumerate}

%19
\item If $\mydet{x+1 & x & x \\
	x & x+\lambda & x \\
	x & x & x+\lambda^2} = \frac{9}{8}\brak{103x+81}$, then $\lambda,\frac{\lambda}{3}$ are the roots of the equation
\begin{enumerate}
\item $4x^2-24x-27=0$            
\item $4x^2+24x+27=0$ 
\item $4x^2-24x+27=0$
\item $4x^2+24x-27=0$
\end{enumerate}

%20
\item The domain of the function $f\brak{x}=\frac{1}{\sqrt{\sbrak{x}^2-3\sbrak{x}-10}}$ is $\brak{\text{where} \sbrak{x} \text{denotes the greatest integer less than or equal to x}}$
\begin{enumerate}
\item $\lbrak{-\infty},\rsbrak{-3} \cup \lsbrak{6},\rbrak{\infty}$             
\item $\lbrak{-\infty},\rsbrak{-2} \cup \lsbrak{5},\rbrak{\infty}$
\item $\lbrak{-\infty},\rsbrak{-3} \cup \lsbrak{5},\rbrak{\infty}$
\item $\lbrak{-\infty},\rsbrak{-2} \cup \lsbrak{6},\rbrak{\infty}$
\end{enumerate}
\end{enumerate}
\begin{enumerate}
\section{SECTION-B}
%1
\item If A is the area in the first quadrant enclosed by the curve C : $2x^2-y+1=0$, the tangent to C at the point $\brak{1,3}$ and the line $x+y=1$, then the value of 60A is

%2
\item Let $A = \cbrak{1,2,3,4,5}$ and $B = \cbrak{1,2,3,4,5,6}$. Then the number of functions $f : A \rightarrow B$ satisfying $f\brak{1}+f\brak{2} = f\brak{4}-1$ is equal to

%3
\item Let the tangent to the parabola $y^2 = 12x$ at the point $\brak{3,\alpha}$ be perpendicular to the line $2x+2y=3$. Then the square of distance of the point $\brak{6,-4}$ from the normal to the hyperbola $\alpha^2x^2 - 9y^2 = 9\alpha^2$ at its point $\brak{\alpha-1,\alpha+2}$ is equal to

%4
\item For $k \in N$, if the sum of the series $1+\frac{4}{k}+\frac{8}{k^2}+\frac{13}{k^3}+\frac{19}{k^4}+ ...$ is 10, then the value of k is

%5
\item Let the line $l : x = \frac{1-y}{-2} = \frac{z-3}{\lambda}, \lambda \in R$ meet the plane $P : x+2y+3z=4$ at the point $\brak{\alpha,\beta,\gamma}$. If the angle between the line l and the plane P is $\cos^{-1}\brak{\sqrt{\frac{5}{14}}}$, then $\alpha+2\beta+6\gamma$ is equal to

%6
\item The number of points where the curve $f\brak{x}  = e^{8x}-e^{6x}-3e^{4x}-e^{2x}+1, x \in R$ cuts x-axis, is equal to

%7
\item If the line $l_1 : 3y-2x=3$ is the angular bisector of the line $l_2 : x-y+1=0$ and $l_3 : \alpha x+\beta y+17$, then $\alpha^2+\beta^2-\alpha-\beta$ is equal to

%8
\item Let the probability of getting head for a biased coin be $\frac{1}{4}$. It is tossed repeatedly until a head appears. Let N be the number of tosses required. If the probability that the equation $64x^2+5Nx+1=0$ has no real root is $\frac{p}{q}$, where p and q are co-prime, then $q-p$ is equal to

%9
\item Let $\vec{a} = \vec{i}+2\vec{j}+3\vec{k}$ and $\vec{b} = \vec{i}+\vec{j}-\vec{k}$. If $\vec{c}$ is a vector such that $\vec{a}\cdot\vec{c}=11$, $\vec{b}\cdot\brak{\vec{a}\times\vec{c}}=27$ and $\vec{b}\cdot\vec{c}=-\sqrt{3}|\vec{b}|$, then $|\vec{a}\times\vec{c}|^2$ is equal to

%10
\item Let $S = \cbrak{z \in C-\cbrak{i,2i}:\frac{z^2+8iz-15}{z^2-3iz-2} \in R}$. If $\alpha-\frac{13}{11}i \in S$, then $242\alpha^2$ is equal to

\end{enumerate}

\end{document}
