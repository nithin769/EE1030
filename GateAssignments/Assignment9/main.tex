\let\negmedspace\undefined
\let\negthickspace\undefined
\documentclass[journal]{IEEEtran}
\usepackage[a5paper, margin=10mm, onecolumn]{geometry}
%\usepackage{lmodern} % Ensure lmodern is loaded for pdflatex
\usepackage{tfrupee} % Include tfrupee package

\setlength{\headheight}{1cm} % Set the height of the header box
\setlength{\headsep}{0mm}     % Set the distance between the header box and the top of the text

\usepackage{gvv-book}
\usepackage{gvv}
\usepackage{cite}
\usepackage{amsmath,amssymb,amsfonts,amsthm}
\usepackage{algorithmic}
\usepackage{graphicx}
\usepackage{textcomp}
\usepackage{xcolor}
\usepackage{txfonts}
\usepackage{listings}
\usepackage{enumitem}
\usepackage{mathtools}
\usepackage{gensymb}
\usepackage{comment}
\usepackage[breaklinks=true]{hyperref}
\usepackage{tkz-euclide} 
\usepackage{listings}
% \usepackage{gvv}                                        
\def\inputGnumericTable{}                                 
\usepackage[latin1]{inputenc}                                
\usepackage{color}                                            
\usepackage{array}                                            
\usepackage{longtable}                                       
\usepackage{calc}                                             
\usepackage{multirow}                                         
\usepackage{hhline}                                           
\usepackage{ifthen}                                           
\usepackage{lscape}
\renewcommand{\thefigure}{\theenumi}
\renewcommand{\thetable}{\theenumi}
\setlength{\intextsep}{10pt} % Space between text and floats


\numberwithin{equation}{enumi}
\numberwithin{figure}{enumi}
\renewcommand{\thetable}{\theenumi}

% Marks the beginning of the document
\begin{document}
\bibliographystyle{IEEEtran}

\title{GateAssignment9}
\author{EE24BTECH11048-NITHIN.K} 
% \maketitle
% \newpage
% \bigskip
{\let\newpage\relax\maketitle}
\begin{enumerate}
\section{Q.36 to Q.65 carry TWO marks Each}
%1
\item The figure shows a thin-walled open-top cylindrical vessel of radius r and wall thickness t. The vessel is held along the brim and contains a constant-density liquid to height h from the base. Neglect atmospheric pressure, the weight of the vessel and bending stresses in the vessel walls. \\
	Which one of the plots depicts qualitatively CORRECT dependence of the magnitudes of axial wall stress $\brak{\sigma_1}$ and circumferential wall stress $\brak{\sigma_2}$ on y?
		\begin{figure}[H]
			\centering
			\resizebox{0.4\textwidth}{!}{%
				\begin{circuitikz}
					\tikzstyle{every node}=[font=\normalsize]
					\draw [line width=2pt, short] (3.25,13.5) -- (4,13.5);
					\draw [short] (4,13.5) -- (4,10.25);
					\draw [short] (4,10.25) -- (6.75,10.25);
					\draw [short] (6.75,10.25) -- (6.75,13.5);
					\draw [line width=2pt, short] (6.75,13.5) -- (7.5,13.5);
					\draw [short] (3.25,13.5) -- (3,13.25);
					\draw [short] (3.5,13.5) -- (3.25,13.25);
					\draw [short] (3.75,13.5) -- (3.5,13.25);
					\draw [short] (4,13.5) -- (3.75,13.25);
					\draw [short] (6.75,13.5) -- (7,13.25);
					\draw [short] (7,13.5) -- (7.25,13.25);
					\draw [short] (7.25,13.5) -- (7.5,13.25);
					\draw [short] (4,13) -- (6.75,13);
					\draw [short] (4.25,12.75) -- (5,12.75);
					\draw [short] (5.5,12.75) -- (6.5,12.75);
					\draw [short] (5,12.25) -- (5.75,12.25);
					\draw [short] (4.5,12.5) -- (5.25,12.5);
					\draw [short] (5.25,11.75) -- (5.5,11.75);
					\draw [short] (6,12.25) -- (6.25,12.25);
					\draw [short] (4.25,12) -- (4.5,12);
					\draw [short] (4.5,11.5) -- (5.25,11.5);
					\draw [short] (5.75,11.5) -- (6.25,11.5);
					\draw [short] (4.25,11.25) -- (5,11.25);
					\draw [short] (5.75,11.75) -- (6.5,11.75);
					\draw [short] (4.75,12) -- (5.25,12);
					\draw [short] (5.75,10.75) -- (6.5,10.75);
					\draw [short] (4.25,10.75) -- (4.75,10.75);
					\draw [short] (5.25,10.5) -- (5.75,10.5);
					\draw [dashed] (5.25,14) -- (5.25,10.25);
					\draw [<->, >=Stealth] (5.25,11) -- (6.75,11);
					\node [font=\normalsize] at (6,11.25) {r};
					\draw [short] (2.75,13) -- (3.75,13);
					\draw [short] (7,13) -- (8,13);
					\draw [short] (7,10.25) -- (8,10.25);
					\draw [->, >=Stealth] (3.25,13) -- (3.25,11.75);
					\node [font=\normalsize] at (3.25,11.5) {y};
					\draw [<->, >=Stealth] (7.5,13) -- (7.5,10.25);
					\node [font=\normalsize] at (7.75,11.5) {h};
					\draw [->, >=Stealth] (3.5,10.5) -- (4,10.5);
					\draw [->, >=Stealth] (4.5,10.5) -- (4,10.5);
					\node [font=\normalsize] at (3.75,10.75) {t};
				\end{circuitikz}
				}%
		\end{figure}
		\begin{enumerate}
			\item
				\begin{figure}[H]
					\resizebox{0.3\textwidth}{!}{%
						\begin{circuitikz}
							\tikzstyle{every node}=[font=\normalsize]
							\draw [line width=0.2pt, ->, >=Stealth] (4,12.25) -- (4,15.25);
							\draw [line width=0.2pt, ->, >=Stealth] (3.5,12.75) -- (6.5,12.75);
							\node [font=\small, rotate around={90:(0,0)}] at (3.75,13.75) {stress};
							\node [font=\normalsize] at (7,12.75) {y};
							\node [font=\normalsize] at (3.5,12.5) {0};
							\draw [short] (4,12.75) -- (6,15);
							\draw [dashed] (6,15) -- (6,12.75);
							\draw [dashed] (4,13.75) -- (6,13.75);
							\draw [dashed] (5,13.75) -- (5,12.75);
							\node [font=\small] at (5,12.5) {h/2};
							\node [font=\small] at (6,12.5) {h};
							\node [font=\normalsize] at (4.5,14) {$\sigma_1$};
							\node [font=\normalsize] at (5.25,14.75) {$\sigma_2$};
						\end{circuitikz}
						}%
				\end{figure}
			\item
				\begin{figure}[H]
					\resizebox{0.3\textwidth}{!}{%
						\begin{circuitikz}    
							\tikzstyle{every node}=[font=\normalsize]
							\draw [line width=0.2pt, ->, >=Stealth] (4,12.25) -- (4,15.25);
							\draw [line width=0.2pt, ->, >=Stealth] (3.5,12.75) -- (6.5,12.75);
							\node [font=\small, rotate around={90:(0,0)}] at (3.75,13.75) {stress};
							\node [font=\normalsize] at (7,12.75) {y};
							\node [font=\normalsize] at (3.5,12.5) {0};
							\draw [short] (4,12.75) -- (6,15);
							\draw [dashed] (6,15) -- (6,12.75);
							\draw [dashed] (4,13.75) -- (6,13.75);
							\draw [dashed] (5,13.75) -- (5,12.75);
							\node [font=\small] at (5,12.5) {h/2};
							\node [font=\small] at (6,12.5) {h};  
							\node [font=\normalsize] at (4.5,14) {$\sigma_2$};
							\node [font=\normalsize] at (5.25,14.75) {$\sigma_1$};
						\end{circuitikz}
						}%
				\end{figure}
			\item
				\begin{figure}[H]
					\resizebox{0.3\textwidth}{!}{%
						\begin{circuitikz}
							\tikzstyle{every node}=[font=\normalsize]
							\draw [line width=0.2pt, ->, >=Stealth] (4,12.25) -- (4,15.25);
							\draw [line width=0.2pt, ->, >=Stealth] (3.5,12.75) -- (6.5,12.75);
							\node [font=\small, rotate around={90:(0,0)}] at (3.75,13.75) {stress};
							\node [font=\normalsize] at (7,12.75) {y};
							\node [font=\normalsize] at (3.75,12.5) {0};
							\draw [dashed] (6,15) -- (6,12.75);
							\draw [dashed] (5,13.75) -- (5,12.75);
							\node [font=\small] at (5,12.5) {h/2};
							\node [font=\small] at (6,12.5) {h};
							\node [font=\normalsize] at (5.25,14.75) {$\sigma_1$};
							\node [font=\normalsize] at (5.75,13.25) {$\sigma_2$};
							\draw [dashed] (4,12.75) -- (6,15);
							\draw [short] (4,12.75) -- (6,13.5);
						\end{circuitikz}
						}%
				\end{figure}
			\item
				\begin{figure}[H]
					\resizebox{0.3\textwidth}{!}{%
						\begin{circuitikz}
							\tikzstyle{every node}=[font=\normalsize]
							\draw [line width=0.2pt, ->, >=Stealth] (4,12.25) -- (4,15.25);
							\draw [line width=0.2pt, ->, >=Stealth] (3.5,12.75) -- (6.5,12.75);
							\node [font=\small, rotate around={90:(0,0)}] at (3.75,13.75) {stress};
							\node [font=\normalsize] at (7,12.75) {y};
							\node [font=\normalsize] at (3.75,12.5) {0};
							\draw [dashed] (6,15) -- (6,12.75);
							\draw [dashed] (5,13.75) -- (5,12.75);                                                  
							\node [font=\small] at (5,12.5) {h/2};
							\node [font=\small] at (6,12.5) {h};
							\node [font=\normalsize] at (5.25,14.75) {$\sigma_2$};
							\node [font=\normalsize] at (5.75,13.25) {$\sigma_1$};
							\draw [dashed] (4,12.75) -- (6,15);
							\draw [short] (4,12.75) -- (6,13.5);
						\end{circuitikz}  
						}%
				\end{figure}
		\end{enumerate}
%2
\item Which one of the following statements is FALSE?
		\begin{enumerate}
			\item For an ideal gas, the enthalpy is independent of pressure.
			\item For a real gas going through an adiabatic reversible process, the process equation is given by $PV^{\gamma}$ = constant, where P is the pressure, V is the volume and $\gamma$ is the 
				ratio of the specific heats of the gas at constant pressure and constant volume.
			\item For an ideal gas undergoing a reversible polytropic process $PV^{1.5}$ = constant, the equation connecting the pressure, volume and temperature of the gas at any point along the pro
				cess is $\frac{P}{R} = \frac{mT}{V}$, where R is the gas constant and m is the mass of the gas.
			\item Any real gas behaves as an ideal gas at sufficiently low pressure or sufficiently high temperature.
		\end{enumerate}
%3
\item Consider a fully adiabatic piston-cylinder arrangement as shown in the figure. The piston is massless and cross-sectional area of the cylinder is A. The fluid inside the cylinder is air $\brak{\text{consid
	ered as a perfect gas}}$, with $\gamma$ being the ratio of the specific heat at constant pressure to the specific heat at constant volume for air. The piston is initially located at a position $L_1$. The 
	initial pressure of the air inside the cylinder is $P_1 >> P_0$, where $P_0$ is the atmospheric pressure. The stop $S_1$ is instantaneously removed and the piston moves to the position $L_2$, where the e
	quilibrium pressure of air inside the cylinder is $P_2 >> P_0$. \\
	What is the work done by the piston on the atmosphere during this process?
		\begin{figure}[H]
			\centering
			\resizebox{0.5\textwidth}{!}{%
				\begin{circuitikz}
					\tikzstyle{every node}=[font=\normalsize]
					\draw [ line width=1.1pt ] (2.25,12.75) rectangle (6,11.5);
					\draw [ fill={rgb,255:red,0; green,0; blue,0} , line width=1.1pt ] (6,12.75) rectangle (6.25,11.5);
					\draw [ line width=1.1pt ] (6.25,12.75) rectangle (8,11.5);
					\draw [ fill={rgb,255:red,0; green,0; blue,0} , line width=1.1pt ] (8,12.75) rectangle (8.25,11.5);
					\node [font=\normalsize] at (4.25,12) {$Air, pressure P_1$};
					\draw [short] (2.25,13.5) -- (2.25,12.75);
					\draw [short] (6,13.5) -- (6,12.75);
					\draw [short] (2.25,11.5) -- (2.25,10.75);
					\draw [short] (8.25,11.5) -- (8.25,10.75);
					\draw [<->, >=Stealth] (2.25,11) -- (8.25,11);
					\draw [<->, >=Stealth] (2.25,13.25) -- (6,13.25);
					\node [font=\normalsize] at (4,13.5) {$L_1$};
					\node [font=\normalsize] at (4.75,10.75) {$L_2$};
					\draw [short] (6.25,12.75) .. controls (6.25,13.5) and (6,13.75) .. (6.5,13.75);
					\draw [short] (6.5,13.75) .. controls (7,14) and (7,14.25) .. (6.75,14.5);
					\draw [short] (6.25,12.75) -- (6.5,13);
					\draw [short] (6.25,12.75) -- (6,13);
					\draw [short] (8.25,12.75) -- (8.5,13);
					\draw [line width=1.1pt, short] (8.25,12.75) -- (9.25,12.75);
					\draw [line width=1.1pt, short] (8.25,11.5) -- (9.25,11.5);
					\draw [short] (8.25,13.5) .. controls (9,13.75) and (9,14) .. (8.75,14.25);
					\draw [short] (8.25,12.75) .. controls (8,13.25) and (7.75,13.5) .. (8.25,13.5);
					\draw [short] (8.25,12.75) -- (7.75,13);
					\node [font=\normalsize] at (6.25,15.25) {Initial position};
					\node [font=\normalsize] at (9.5,15) {Final position of};
					\node [font=\normalsize] at (6.25,14.75) {of the piston};
					\node [font=\normalsize] at (9.25,14.5) {the piston};
					\node [font=\normalsize] at (10.5,12.25) {Atmosphere};
					\node [font=\normalsize] at (10.5,11.75) {$pressure P_0$};
					\draw [ fill={rgb,255:red,255; green,190; blue,111} , line width=0.8pt ] (6.25,12) rectangle (7,11.5);
					\draw [ fill={rgb,255:red,255; green,190; blue,111} , line width=0.8pt ] (8.25,12) rectangle (9,11.5);
					\node [font=\normalsize] at (6.5,11.75) {$S_1$};
					\node [font=\normalsize] at (8.5,11.75) {$S_2$};
				\end{circuitikz}
				}%
		\end{figure}
		\begin{enumerate}
			\item 0
			\item $P_0A\brak{L_2 - L_1}$
			\item $P_1AL_1\ln\frac{L_1}{L_2}$
			\item $\frac{\brak{P_2L_2 - P_1L_1}}{\brak{1 - \gamma}}$
		\end{enumerate}
%4
\item A cylindrical rod of length h and diameter d is placed inside a cubic enclosure of side length L. S denotes the inner surface of the cube. The view-factor $F_{S-S}$ is
	\begin{enumerate}
                \item 0
                \item 1
		\item $\frac{\brak{\pi dh + \pi d^2/2}}{6L^2}$
                \item $1 - \frac{\brak{\pi dh + \pi d^2/2}}{6L^2}$
	\end{enumerate}
%5
\item In an ideal orthogonal cutting experiment $\brak{\text{see figure}}$, the cutting speed V is 1 m/s, the rake angle of the tool $\alpha$ = 5$\degree$, and the shear angle, $\phi$, is known to be 45$\degree$
	. \\
	Applying the ideal orthogonal cutting model, consider two shear planes PQ and RS close to each other. As they approach the thin shear zone $\brak{\text{shown as a thick line in the figure}}$, plane RS ge
	ts sheared with respect to PQ $\brak{\text{point R1 shears to R2, and S1 shears to S2}}$. \\
	Assuming that the perpendicular distance between PQ and RS is $\delta = 25 \mu m$, what is the value of shear strain rate $\brak{\text{in } s^{-1}}$ that the material undergoes at the shear zone?
	\begin{figure}[H]
		\centering
		\resizebox{0.6\textwidth}{!}{%
			\begin{circuitikz}
				\tikzstyle{every node}=[font=\normalsize]
				\draw [dashed] (3.25,15.5) -- (3.25,12.25);
				\node at (3.25,13.25) [circ] {};
				\draw [short] (3.25,13.25) -- (5,13.25);
				\draw [short] (3.25,13.25) -- (3.75,15);
				\draw [short] (3.75,15) -- (5,13.25);
				\draw [short] (3.25,14.5) .. controls (3.25,15) and (3.75,14.75) .. (3.5,14.25);
				\draw [dashed] (3.25,13.25) -- (4.5,14.25);
				\node at (5,13.25) [circ] {};
				\node at (3.75,15) [circ] {};
				\draw [<->, >=Stealth] (4,15.25) -- (5.25,13.5);
				\node [font=\normalsize] at (4.5,13.5) {$\varphi$};
				\node [font=\normalsize] at (4.25,13.75) {$\delta$};
				\node [font=\normalsize] at (3.5,14.75) {$\alpha$};
				\node [font=\normalsize] at (2.75,13.25) {P1};
				\node [font=\normalsize] at (5.5,13) {S1};
				\node [font=\normalsize] at (3.75,15.5) {S2};
				\draw [ line width=0.5pt , rotate around={38:(4.125, 14.125)}] (4.25,14.25) rectangle (4,14);
				\draw [ line width=0.8pt , dashed] (9.25,11.25) circle (1.75cm);
				\draw [line width=0.7pt, short] (8.75,10.75) -- (10,15.25);
				\draw [line width=0.7pt, dashed] (8.75,10.75) -- (10,10.75);
				\draw [line width=0.7pt, dashed] (9,12) -- (12.25,8.25);
				\node at (8.75,10.75) [circ] {};
				\node at (9,12) [circ] {};
				\node at (10,10.75) [circ] {};
				\draw [line width=0.7pt, short] (8.75,10.75) -- (5.5,10.75);
				\node at (7.25,10.75) [circ] {};
				\draw [line width=0.9pt, short] (5.5,10.75) .. controls (6.5,8.5) and (4,9.5) .. (5.5,7.5);
				\draw [line width=0.8pt, short] (5.5,7.5) -- (5.25,6);
				\draw [line width=0.8pt, short] (5.25,6) -- (17.25,6);
				\draw [line width=0.8pt, dashed] (5.25,8) -- (14,8.25);
				\draw [line width=0.8pt, dashed] (7.25,10.75) -- (9.25,8.25);
				\node at (9.25,8.25) [circ] {};
				\node at (12.25,8.25) [circ] {};
				\draw [line width=2pt, short] (8.75,10.75) -- (10.75,8.25);
				\draw [line width=0.8pt, dashed] (10.75,8.25) -- (11.75,13.5);
				\node at (10.75,8.25) [circ] {};
				\draw [ fill={rgb,255:red,192; green,191; blue,188} , line width=0.2pt ] (7.25,10.75) -- (8.75,10.75) -- (10.75,8.25) -- (9.25,8.25) -- cycle;
				\draw [line width=0.8pt, short] (8.25,8) .. controls (7.75,8.5) and (8.25,9) .. (8.75,9);
				\draw [line width=0.8pt, short] (10.75,8.25) -- (13.75,15);
				\draw [line width=0.8pt, short] (13.75,15) -- (16.5,15);
				\draw [line width=0.8pt, short] (16.5,15) -- (16.5,10.25);
				\draw [line width=0.8pt, short] (16.5,10.25) -- (10.75,8.25);
				\draw [line width=0.8pt, short] (11.5,12.25) .. controls (12,12.75) and (12.5,12.5) .. (12.25,11.75);
				\draw [line width=0.8pt, short] (10,15.25) .. controls (10.5,14.75) and (11,16.25) .. (11.25,15);
				\draw [line width=0.8pt, short] (11.25,15) .. controls (12,14.25) and (12.25,15.5) .. (13.5,14.25);
				\node [font=\large] at (7,11) {P};
				\node [font=\large] at (12,12) {$\alpha$};
				\node [font=\large] at (8.5,12) {S2};
				\node [font=\large] at (8.25,11) {S};
				\node [font=\large] at (10.5,11) {S1};
				\node [font=\large] at (14.5,11.5) {stationary tool};
				\node [font=\large] at (8.5,8.5) {$\varphi$};
				\node [font=\large] at (9.25,7.75) {Q};
				\node [font=\large] at (10.75,7.5) {R};
				\node [font=\large] at (12.75,7.75) {R1};
				\node at (11.25,9.25) [circ] {};
				\draw [line width=0.7pt, ->, >=Stealth] (12.5,8.5) -- (11.75,9.25);
				\node [font=\large] at (11.75,9.5) {R2};
				\draw [line width=0.7pt, short] (13.75,8.25) -- (17,8.25);
				\draw [line width=1.1pt, short] (17,8.25) .. controls (16,7.75) and (17.75,7.25) .. (17.25,6);
				\draw [line width=0.7pt, ->, >=Stealth] (6.5,6.75) -- (8.5,6.75);
				\node [font=\large] at (7.25,7.25) {V};
				\node [font=\normalsize] at (9.75,6.5) {Work material is moving with a speed V};
				\draw [line width=1.1pt, short] (8.25,12.5) -- (7.25,13);
				\draw [line width=1.1pt, short] (8.25,12.5) -- (7.75,13.5);
				\draw [line width=1.1pt, short] (6.75,12.5) -- (7.25,13);
				\draw [line width=1.1pt, short] (7.75,13.5) -- (8.25,14);
				\draw [line width=1.1pt, short] (6.75,12.5) -- (6.5,14);
				\draw [line width=1.1pt, short] (6.5,14) -- (8.25,14);
			\end{circuitikz}
			}%
		\end{figure}
		\begin{enumerate}
			\item $1.84 \times 10^4$
			\item $5.20 \times 10^4$
			\item $0.71 \times 10^4$
			\item $1.30 \times 10^4$
		\end{enumerate}
%6
\item A CNC machine has one of its linear positioning axes as shown in the figure, consisting of a motor rotating a lead screw, which in turn moves a nut horizontally on which a table is mounted. The motor moves in discrete rotational steps of 50 steps per revolution. The pitch of the screw is 5 mm and the total horizontal traverse length of the table is 100 mm. What is the total number of controllable locations at which the table can be positioned on this axis?
	\begin{figure}[H]
		\centering
		\resizebox{0.5\textwidth}{!}{%
			\begin{circuitikz}
				\tikzstyle{every node}=[font=\large]
				\draw [line width=2pt, short] (-2,10.75) -- (11.25,10.75);
				\draw [ fill={rgb,255:red,192; green,191; blue,188} , line width=0.2pt ] (-2,10.75) rectangle (11,9.75);
				\draw [ fill={rgb,255:red,0; green,0; blue,0} , line width=1.1pt ] (9.5,12.75) rectangle (10.5,10.75);
				\draw [ line width=1.1pt ] (-1.5,12.75) rectangle (1.25,10.75);
				\draw [ line width=1.1pt ] (1.25,12.25) rectangle (1.75,11.25);
				\draw [ line width=1.1pt ] (1.75,12) rectangle (4.75,11.5);
				\draw [ line width=1.1pt ] (4.75,12.5) rectangle (6.25,11.25);
				\draw [ line width=1.1pt ] (6.25,12) rectangle (9.75,11.5);
				\draw [ fill={rgb,255:red,192; green,191; blue,188} , line width=1.1pt ] (4,13) rectangle (7,12.5);
				\node [font=\large] at (-0.25,11.75) {Motor};
				\node [font=\normalsize] at (2.75,11.25) {screw};
				\node [font=\large] at (5.5,11.75) {Nut};
				\node [font=\large] at (6.5,13.5) {Table};
				\draw [line width=0.7pt, <->, >=Stealth] (4.25,13.5) -- (5.75,13.5);
				\draw [line width=0.8pt, ->, >=Stealth] (-2,14.5) -- (-1.5,13);
				\node [font=\large] at (-2.25,15.75) {Motor that};
				\node [font=\large] at (-2.25,15.25) {rotates in};
				\node [font=\large] at (-1.75,14.75) {discrete steps};
				\draw [line width=1.1pt, short] (2,12) -- (1.75,11.5);
				\draw [line width=1.1pt, short] (2,11.5) -- (2.25,12);
				\draw [line width=1.1pt, short] (2.25,11.5) -- (2.5,12);
				\draw [line width=1.1pt, short] (2.75,11.5) -- (3,12);
				\draw [line width=1.1pt, short] (2.5,11.5) -- (2.75,12);
				\draw [line width=1.1pt, short] (3.25,11.5) -- (3.5,12);
				\draw [line width=1.1pt, short] (3,11.5) -- (3.25,12);
				\draw [line width=1.1pt, short] (3.75,11.5) -- (4,12);
				\draw [line width=1.1pt, short] (4,11.5) -- (4.25,12);
				\draw [line width=1.1pt, short] (4.25,11.5) -- (4.5,12);
				\draw [line width=1.1pt, short] (3.5,11.5) -- (3.75,12);
				\draw [line width=1.1pt, short] (6.25,11.75) -- (6.5,12);
				\draw [line width=1.1pt, short] (6.5,11.5) -- (7,12);
				\draw [line width=1.1pt, short] (6.75,11.5) -- (7.25,12);
				\draw [line width=1.1pt, short] (7.25,11.5) -- (7.75,12);
				\draw [line width=1.1pt, short] (7,11.5) -- (7.5,12);
				\draw [line width=1.1pt, short] (7.5,11.5) -- (8,12);
				\draw [line width=1.1pt, short] (8,11.5) -- (8.5,12);
				\draw [line width=1.1pt, short] (7.75,11.5) -- (8.25,12);
				\draw [line width=1.1pt, short] (8.5,11.5) -- (9,12);
				\draw [line width=1.1pt, short] (8.25,11.5) -- (8.75,12);
				\draw [line width=1.1pt, short] (8.75,11.5) -- (9.25,12);
				\draw [line width=1.1pt, short] (4.5,11.5) -- (4.75,12);
				\draw [line width=1.1pt, short] (6.25,11.5) -- (6.75,12);
				\draw [line width=1.1pt, short] (9,11.5) -- (9.5,12);
				\draw [line width=1.1pt, short] (9.25,11.5) -- (9.5,11.75);
				\draw [line width=1.1pt, ->, >=Stealth] (8,11.25) .. controls (9,12) and (8.75,13.25) .. (8,12.25) ;
			\end{circuitikz}
			}%
	\end{figure}
	\begin{enumerate}
		\item 5000
		\item 2
		\item 1000
		\item 200
	\end{enumerate}
%7
\item Cylindrical bars P and Q have identical lengths and radii, but are composed of different linear elastic materials. The Young\'s modulus and coefficient of thermal expansion of Q are twice the corresponding 
	values of P. Assume the bars to be perfectly bonded at the interface, and their weights to be negligible. \\
	The bars are held between rigid supports as shown in the figure and the temperature is raised by $\Delta T$. Assume that the stress in each bar is homogeneous and uniaxial. Denote the magnitudes of stres
	s in P and Q by $\sigma_1$ and $\sigma_2$ , respectively. \\
	Which of the statement(s) given is/are CORRECT?
	\begin{figure}[H]
		\centering
		\resizebox{0.5\textwidth}{!}{%
			\begin{circuitikz}
				\tikzstyle{every node}=[font=\Large]
				\draw [ line width=0.2pt ] (4.5,19.5) rectangle (5.25,15.75);
				\draw [ line width=0.2pt ] (11.25,19.5) rectangle (12,15.75);
				\draw [ fill={rgb,255:red,192; green,191; blue,188} , line width=1.1pt ] (5.25,18.25) rectangle (8.25,17);
				\draw [ line width=1.1pt ] (8.25,18.25) rectangle (11.25,17);
				\node [font=\Large] at (6.75,17.5) {$\sigma_1$};
				\node [font=\Large] at (9.75,17.5) {$\sigma_2$};
				\node [font=\large] at (9.75,18.75) {Q};
				\node [font=\large] at (6.75,18.75) {P};
				\draw [line width=0.8pt, short] (4.5,19.25) -- (4.75,19.5);
				\draw [line width=0.8pt, short] (4.5,19) -- (5,19.5);
				\draw [line width=0.8pt, short] (4.5,18.75) -- (5.25,19.5);
				\draw [line width=0.8pt, short] (4.5,18.5) -- (5.25,19.25);
				\draw [line width=0.8pt, short] (4.5,18.25) -- (5.25,19);
				\draw [line width=0.8pt, short] (4.5,18) -- (5.25,18.75);
				\draw [line width=0.8pt, short] (4.5,17.75) -- (5.25,18.5);
				\draw [line width=0.8pt, short] (4.5,17.5) -- (5.25,18.25);
				\draw [line width=0.8pt, short] (4.5,17.25) -- (5.25,18);
				\draw [line width=0.8pt, short] (4.5,17) -- (5.25,17.75);
				\draw [line width=0.8pt, short] (4.5,16.75) -- (5.25,17.5);
				\draw [line width=0.8pt, short] (4.5,16.5) -- (5.25,17.25);
				\draw [line width=0.8pt, short] (4.5,16.25) -- (5.25,17);
				\draw [line width=0.8pt, short] (4.5,16) -- (5.25,16.75);
				\draw [line width=0.8pt, short] (4.75,15.75) -- (5.25,16.25);
				\draw [line width=0.8pt, short] (4.5,15.75) -- (5.25,16.5);
				\draw [line width=0.8pt, short] (5,15.75) -- (5.25,16);
				\draw [line width=0.8pt, short] (11.25,19.5) -- (12,19);
				\draw [line width=0.8pt, short] (11.5,19.5) -- (12,19.25);
				\draw [line width=0.8pt, short] (11.25,19.25) -- (12,18.75);
				\draw [line width=0.8pt, short] (11.25,19) -- (12,18.5);
				\draw [line width=0.8pt, short] (11.25,18.75) -- (12,18.25);
				\draw [line width=0.8pt, short] (11.25,18.5) -- (12,18);
				\draw [line width=0.8pt, short] (11.25,18.25) -- (12,17.75);
				\draw [line width=0.8pt, short] (11.25,18) -- (12,17.5);
				\draw [line width=0.8pt, short] (11.25,17.5) -- (12,17);
				\draw [line width=0.8pt, short] (11.25,17.75) -- (12,17.25);
				\draw [line width=0.8pt, short] (11.25,17.25) -- (12,16.75);
				\draw [line width=0.8pt, short] (11.25,17) -- (12,16.5);
				\draw [line width=0.8pt, short] (11.25,16.75) -- (12,16.25);
				\draw [line width=0.8pt, short] (11.25,16.5) -- (12,16);
				\draw [line width=0.8pt, short] (11.25,16.25) -- (12,15.75);
				\draw [line width=0.8pt, short] (11.25,16) -- (11.75,15.75);
			\end{circuitikz}
			}%
	\end{figure}
	\begin{enumerate}
		\item The interface between P and Q moves to the left after heating
		\item The interface between P and Q moves to the right after heating
		\item $\sigma_1 < \sigma_2$
		\item $\sigma_1 = \sigma_2$
	\end{enumerate}
%8
\item A very large metal plate of thickness d and thermal conductivity k is cooled by a stream of air at temperature $T_{\infty}$ = 300 K with a heat transfer coefficient h, as shown in the figure. The centerlin
	e temperature of the plate is $T_P$. In which of the following case(s) can the lumped parameter model be used to study the heat transfer in the metal plate?
	\begin{figure}[H]
		\centering
		\resizebox{0.5\textwidth}{!}{%
			\begin{circuitikz}
				\tikzstyle{every node}=[font=\Large]
				\draw [line width=0.8pt, short] (9.5,18.75) -- (17.5,18.75);
				\draw [line width=0.8pt, short] (9.5,17.25) -- (17.5,17.25);
				\draw [line width=0.8pt, dashed] (8.5,18.75) -- (9.5,18.75);
				\draw [line width=0.8pt, dashed] (8.5,17.25) -- (9.5,17.25);
				\draw [line width=0.8pt, dashed] (17.5,18.75) -- (18.5,18.75);
				\draw [line width=0.8pt, dashed] (17.5,17.25) -- (18.5,17.25);
				\draw [line width=0.5pt, dashed] (9.25,18) .. controls (13.75,18) and (13.5,18) .. (17.75,18);
				\draw [line width=0.6pt, <->, >=Stealth] (9.25,18.75) -- (9.25,17.25);
				\draw [line width=0.6pt, <->, >=Stealth] (17.5,18) -- (17.5,17.25);
				\draw [line width=0.8pt, ->, >=Stealth] (12,19.5) -- (14.75,19.5);
				\draw [line width=0.8pt, ->, >=Stealth] (12.25,16.5) -- (15,16.5);
				\node [font=\Large] at (13.5,17.75) {$T_P$};
				\node [font=\Large] at (13,20) {$T_{\infty}$};
				\node [font=\large] at (8.5,18) {d};
				\node [font=\large] at (18,17.5) {d/2};
				\node [font=\Large] at (13.5,16) {$T_{\infty}$};
			\end{circuitikz}
			}%
	\end{figure}
	\begin{enumerate}
		\item $h = 10 Wm^{-2}K^{-1}, k = 100 Wm^{-1}K^{-1}, d = 1 mm, T_P = 350 K$
                \item $h = 100 Wm^{-2}K^{-1}, k = 100 Wm^{-1}K^{-1}, d = 1 m, T_P = 325 K$
                \item $h = 100 Wm^{-2}K^{-1}, k = 1000 Wm^{-1}K^{-1}, d = 1 mm, T_P = 325 K$
                \item $h = 1000 Wm^{-2}K^{-1}, k = 1 Wm^{-1}K^{-1}, d = 1 m, T_P = 350 K$
        \end{enumerate}
%9
\item The smallest perimeter that a rectangle with area of 4 square units can have is \rule{1cm}{0.4pt} units. \\
	$\brak{\text{Answer in integer}}$
%10
\item Consider the second-order linear ordinary differential equation
	\begin{align*}
		x^2\frac{d^2y}{dx^2} + x\frac{dy}{dx} - y = 0, x \geq 1
	\end{align*}
	with the initial conditions
	\begin{align*}
		y\brak{x = 1} = 6, \frac{dy}{dx}\bigg|_{x=1} = 2.
	\end{align*}
	The value of y at x = 2 equals \rule{1cm}{0.4pt}. \\
	$\brak{\text{Answer in integer}}$
%11
\item The initial value problem
	\begin{align*}
		\frac{dy}{dt} + 2y = 0, y\brak{0} = 1
	\end{align*}
	is solved numerically using the forward Euler\'s method with a constant and positive time step of $\Delta t$. \\
	Let $y_n$ represent the numerical solution obtained after n steps. The condition $|y_{n+1} \leq |y_n|$ is satisfied if and only if $\Delta t$ does not exceed \rule{1cm}{0.4pt}. \\
	$\brak{\text{Answer in integer}}$
%12
\item The atomic radius of a hypothetical face-centered cubic (FCC) metal is ($\sqrt{2}$/10) nm. The atomic weight of the metal is 24.092 g/mol. Taking Avogadro\'s number to be $6.023 \times 10^{23}$ atoms/mol, the density of the metal is \rule{1cm}{0.4pt} kg/$m^3$. \\
	$\brak{\text{Answer in integer}}$
%13
\item A steel sample with 1.5 wt.\% carbon $\brak{\text{no other alloying elements present}}$ is slowly cooled from 1100 $\degree$C to just below the eutectoid temperature (723 $\degree$ C). A part of the iron-cementite phase diagram is shown in the figure. The ratio of the pro-eutectoid cementite content to the total cementite content in the microstructure that develops just below the eutectoid temperature is \rule{1cm}{0.4pt}. \\
	$\brak{\text{Rounded off to two decimal places}}$
	\begin{figure}[H]
		\centering
		\resizebox{0.5\textwidth}{!}{%
			\begin{circuitikz}
				\tikzstyle{every node}=[font=\Large]
				\draw [line width=0.7pt, short] (7.25,21.25) -- (7.25,13);
				\draw [line width=0.8pt, short] (7.25,13) -- (16,13);
				\draw [line width=0.8pt, short] (7.5,13) -- (8.5,16.5);
				\draw [line width=0.8pt, short] (8.5,16.5) -- (7.25,18.5);
				\draw [line width=0.8pt, short] (7.25,18.5) -- (11.25,16.5);
				\draw [line width=0.8pt, dashed] (7.25,16.5) -- (8.5,16.5);
				\draw [line width=0.8pt, short] (8.5,16.5) -- (16,16.5);
				\draw [line width=0.8pt, dashed] (8.5,16.25) -- (8.5,13);
				\draw [line width=0.8pt, dashed] (11.25,16.5) -- (11.25,13);
				\draw [line width=0.8pt, short] (11.25,16.5) -- (14,19.5);
				\draw [line width=1.4pt, dashed] (13.25,19.25) -- (13.25,13);
				\node at (13.25,19.25) [circ] {};
				\draw [line width=0.8pt, short] (14,19.5) -- (16,19.5);
				\draw [line width=0.8pt, dashed] (12.75,21) -- (14,19.5);
				\draw [line width=0.8pt, dashed] (14,19.5) -- (14,13);
				\draw [line width=0.8pt, dashed] (16.5,19.5) -- (17.75,19.5);
				\draw [line width=0.8pt, dashed] (16.5,16.5) -- (17.75,16.5);
				\draw [line width=0.8pt, dashed] (16.5,13) -- (17.75,13);
				\draw [line width=0.8pt, short] (17.75,21.75) -- (17.75,13);
				\draw [line width=0.8pt, ->, >=Stealth] (17,14) -- (17.75,13.25);
				\node [font=\normalsize, rotate around={90:(0,0)}] at (6.75,18.75) {Temperature($\circ$ C)};
				\node [font=\normalsize] at (6.25,16.5) {723$\circ$ C};
				\node [font=\large] at (7.75,15.75) {$\alpha$};
				\node [font=\large] at (8.75,12.5) {0.035};
				\node [font=\large] at (11.25,12.5) {0.8};
				\node [font=\large] at (10.75,18.25) {$\gamma$};
				\node [font=\normalsize] at (12.5,19.5) {1100$\circ$ C};
				\node [font=\large] at (16.5,14.5) {$Fe_3C$};
				\node [font=\large] at (18,12.5) {6.7};
				\node [font=\large] at (14,12.5) {1.7};
				\node [font=\large] at (13,12.5) {1.5};
				\node [font=\Large] at (13,11) {wt.\%C};
			\end{circuitikz}
			}%
	\end{figure}
\end{enumerate}
\end{document}
