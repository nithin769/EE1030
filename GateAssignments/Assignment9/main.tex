\let\negmedspace\undefined
\let\negthickspace\undefined
\documentclass[journal]{IEEEtran}
\usepackage[a5paper, margin=10mm, onecolumn]{geometry}
%\usepackage{lmodern} % Ensure lmodern is loaded for pdflatex
\usepackage{tfrupee} % Include tfrupee package

\setlength{\headheight}{1cm} % Set the height of the header box
\setlength{\headsep}{0mm}     % Set the distance between the header box and the top of the text

\usepackage{gvv-book}
\usepackage{gvv}
\usepackage{cite}
\usepackage{amsmath,amssymb,amsfonts,amsthm}
\usepackage{algorithmic}
\usepackage{graphicx}
\usepackage{textcomp}
\usepackage{xcolor}
\usepackage{txfonts}
\usepackage{listings}
\usepackage{enumitem}
\usepackage{mathtools}
\usepackage{gensymb}
\usepackage{comment}
\usepackage[breaklinks=true]{hyperref}
\usepackage{tkz-euclide} 
\usepackage{listings}
% \usepackage{gvv}                                        
\def\inputGnumericTable{}                                 
\usepackage[latin1]{inputenc}                                
\usepackage{color}                                            
\usepackage{array}                                            
\usepackage{longtable}                                       
\usepackage{calc}                                             
\usepackage{multirow}                                         
\usepackage{hhline}                                           
\usepackage{ifthen}                                           
\usepackage{lscape}
\renewcommand{\thefigure}{\theenumi}
\renewcommand{\thetable}{\theenumi}
\setlength{\intextsep}{10pt} % Space between text and floats


\numberwithin{equation}{enumi}
\numberwithin{figure}{enumi}
\renewcommand{\thetable}{\theenumi}

% Marks the beginning of the document
\begin{document}
\bibliographystyle{IEEEtran}

\title{GateAssignment9}
\author{EE24BTECH11048-NITHIN.K} 
% \maketitle
% \newpage
% \bigskip
{\let\newpage\relax\maketitle}
\begin{enumerate}
		\section{Q.36 to Q.65 carry TWO marks Each}
		%1
	\item The figure shows a thin-walled open-top cylindrical vessel of radius r and wall thickness t. The vessel is held along the brim and contains a constant-density liquid to height h from the base. Neglect atmospheric pressure, the weight of the vessel and bending stresses in the vessel walls. \\
		Which one of the plots depicts qualitatively CORRECT dependence of the magnitudes of axial wall stress $\brak{\sigma_1}$ and circumferential wall stress $\brak{\sigma_2}$ on y?
		\begin{figure}[H]
			\centering
			\resizebox{0.4\textwidth}{!}{%
				\begin{circuitikz}
					\tikzstyle{every node}=[font=\normalsize]
					\draw [line width=2pt, short] (3.25,13.5) -- (4,13.5);
					\draw [short] (4,13.5) -- (4,10.25);
					\draw [short] (4,10.25) -- (6.75,10.25);
					\draw [short] (6.75,10.25) -- (6.75,13.5);
					\draw [line width=2pt, short] (6.75,13.5) -- (7.5,13.5);
					\draw [short] (3.25,13.5) -- (3,13.25);
					\draw [short] (3.5,13.5) -- (3.25,13.25);
					\draw [short] (3.75,13.5) -- (3.5,13.25);
					\draw [short] (4,13.5) -- (3.75,13.25);
					\draw [short] (6.75,13.5) -- (7,13.25);
					\draw [short] (7,13.5) -- (7.25,13.25);
					\draw [short] (7.25,13.5) -- (7.5,13.25);
					\draw [short] (4,13) -- (6.75,13);
					\draw [short] (4.25,12.75) -- (5,12.75);
					\draw [short] (5.5,12.75) -- (6.5,12.75);
					\draw [short] (5,12.25) -- (5.75,12.25);
					\draw [short] (4.5,12.5) -- (5.25,12.5);
					\draw [short] (5.25,11.75) -- (5.5,11.75);
					\draw [short] (6,12.25) -- (6.25,12.25);
					\draw [short] (4.25,12) -- (4.5,12);
					\draw [short] (4.5,11.5) -- (5.25,11.5);
					\draw [short] (5.75,11.5) -- (6.25,11.5);
					\draw [short] (4.25,11.25) -- (5,11.25);
					\draw [short] (5.75,11.75) -- (6.5,11.75);
					\draw [short] (4.75,12) -- (5.25,12);
					\draw [short] (5.75,10.75) -- (6.5,10.75);
					\draw [short] (4.25,10.75) -- (4.75,10.75);
					\draw [short] (5.25,10.5) -- (5.75,10.5);
					\draw [dashed] (5.25,14) -- (5.25,10.25);
					\draw [<->, >=Stealth] (5.25,11) -- (6.75,11);
					\node [font=\normalsize] at (6,11.25) {r};
					\draw [short] (2.75,13) -- (3.75,13);
					\draw [short] (7,13) -- (8,13);
					\draw [short] (7,10.25) -- (8,10.25);
					\draw [->, >=Stealth] (3.25,13) -- (3.25,11.75);
					\node [font=\normalsize] at (3.25,11.5) {y};
					\draw [<->, >=Stealth] (7.5,13) -- (7.5,10.25);
					\node [font=\normalsize] at (7.75,11.5) {h};
					\draw [->, >=Stealth] (3.5,10.5) -- (4,10.5);
					\draw [->, >=Stealth] (4.5,10.5) -- (4,10.5);
					\node [font=\normalsize] at (3.75,10.75) {t};
				\end{circuitikz}
				}%
		\end{figure}
		\begin{enumerate}
			\item
				\begin{figure}[H]
					\resizebox{0.3\textwidth}{!}{%
						\begin{circuitikz}
							\tikzstyle{every node}=[font=\normalsize]
							\draw [line width=0.2pt, ->, >=Stealth] (4,12.25) -- (4,15.25);
							\draw [line width=0.2pt, ->, >=Stealth] (3.5,12.75) -- (6.5,12.75);
							\node [font=\small, rotate around={90:(0,0)}] at (3.75,13.75) {stress};
							\node [font=\normalsize] at (7,12.75) {y};
							\node [font=\normalsize] at (3.5,12.5) {0};
							\draw [short] (4,12.75) -- (6,15);
							\draw [dashed] (6,15) -- (6,12.75);
							\draw [dashed] (4,13.75) -- (6,13.75);
							\draw [dashed] (5,13.75) -- (5,12.75);
							\node [font=\small] at (5,12.5) {h/2};
							\node [font=\small] at (6,12.5) {h};
							\node [font=\normalsize] at (4.5,14) {$\sigma_1$};
							\node [font=\normalsize] at (5.25,14.75) {$\sigma_2$};
						\end{circuitikz}
						}%
				\end{figure}
			\item
				\begin{figure}[H]
					\resizebox{0.3\textwidth}{!}{%
						\begin{circuitikz}    
							\tikzstyle{every node}=[font=\normalsize]
							\draw [line width=0.2pt, ->, >=Stealth] (4,12.25) -- (4,15.25);
							\draw [line width=0.2pt, ->, >=Stealth] (3.5,12.75) -- (6.5,12.75);
							\node [font=\small, rotate around={90:(0,0)}] at (3.75,13.75) {stress};
							\node [font=\normalsize] at (7,12.75) {y};
							\node [font=\normalsize] at (3.5,12.5) {0};
							\draw [short] (4,12.75) -- (6,15);
							\draw [dashed] (6,15) -- (6,12.75);
							\draw [dashed] (4,13.75) -- (6,13.75);
							\draw [dashed] (5,13.75) -- (5,12.75);
							\node [font=\small] at (5,12.5) {h/2};
							\node [font=\small] at (6,12.5) {h};  
							\node [font=\normalsize] at (4.5,14) {$\sigma_2$};
							\node [font=\normalsize] at (5.25,14.75) {$\sigma_1$};
						\end{circuitikz}
						}%
				\end{figure}
			\item
				\begin{figure}[H]
					\resizebox{0.3\textwidth}{!}{%
						\begin{circuitikz}
							\tikzstyle{every node}=[font=\normalsize]
							\draw [line width=0.2pt, ->, >=Stealth] (4,12.25) -- (4,15.25);
							\draw [line width=0.2pt, ->, >=Stealth] (3.5,12.75) -- (6.5,12.75);
							\node [font=\small, rotate around={90:(0,0)}] at (3.75,13.75) {stress};
							\node [font=\normalsize] at (7,12.75) {y};
							\node [font=\normalsize] at (3.75,12.5) {0};
							\draw [dashed] (6,15) -- (6,12.75);
							\draw [dashed] (5,13.75) -- (5,12.75);
							\node [font=\small] at (5,12.5) {h/2};
							\node [font=\small] at (6,12.5) {h};
							\node [font=\normalsize] at (5.25,14.75) {$\sigma_1$};
							\node [font=\normalsize] at (5.75,13.25) {$\sigma_2$};
							\draw [dashed] (4,12.75) -- (6,15);
							\draw [short] (4,12.75) -- (6,13.5);
						\end{circuitikz}
						}%
				\end{figure}
			\item
				\begin{figure}[H]
					\resizebox{0.3\textwidth}{!}{%
						\begin{circuitikz}
							\tikzstyle{every node}=[font=\normalsize]
							\draw [line width=0.2pt, ->, >=Stealth] (4,12.25) -- (4,15.25);
							\draw [line width=0.2pt, ->, >=Stealth] (3.5,12.75) -- (6.5,12.75);
							\node [font=\small, rotate around={90:(0,0)}] at (3.75,13.75) {stress};
							\node [font=\normalsize] at (7,12.75) {y};
							\node [font=\normalsize] at (3.75,12.5) {0};
							\draw [dashed] (6,15) -- (6,12.75);
							\draw [dashed] (5,13.75) -- (5,12.75);                                                  
							\node [font=\small] at (5,12.5) {h/2};
							\node [font=\small] at (6,12.5) {h};
							\node [font=\normalsize] at (5.25,14.75) {$\sigma_2$};
							\node [font=\normalsize] at (5.75,13.25) {$\sigma_1$};
							\draw [dashed] (4,12.75) -- (6,15);
							\draw [short] (4,12.75) -- (6,13.5);
						\end{circuitikz}  
						}%
				\end{figure}
		\end{enumerate}
		%2
	\item Which one of the following statements is FALSE?
		\begin{enumerate}
			\item For an ideal gas, the enthalpy is independent of pressure.
			\item For a real gas going through an adiabatic reversible process, the process equation is given by $PV^{\gamma}$ = constant, where P is the pressure, V is the volume and $\gamma$ is the ratio o
				f the specific heats of the gas at constant pressure and constant volume.
			\item For an ideal gas undergoing a reversible polytropic process $PV^{1.5}$ = constant, the equation connecting the pressure, volume and temperature of the gas at any point along the process is
				$\frac{P}{R} = \frac{mT}{V}$, where R is the gas constant and m is the mass of the gas.
			\item Any real gas behaves as an ideal gas at sufficiently low pressure or sufficiently high temperature.
		\end{enumerate}
		%3
	\item Consider a fully adiabatic piston-cylinder arrangement as shown in the figure. The piston is massless and cross-sectional area of the cylinder is A. The fluid inside the cylinder is air $\brak{\text{consid
		ered as a perfect gas}}$, with $\gamma$ being the ratio of the specific heat at constant pressure to the specific heat at constant volume for air. The piston is initially located at a position $L_1$. The
		initial pressure of the air inside the cylinder is $P_1 >> P_0$, where $P_0$ is the atmospheric pressure. The stop $S_1$ is instantaneously removed and the piston moves to the position $L_2$, where the 
		equilibrium pressure of air inside the cylinder is $P_2 >> P_0$. \\
		What is the work done by the piston on the atmosphere during this process?
		\begin{figure}[!ht]
			\centering
			\resizebox{0.5\textwidth}{!}{%
				\begin{circuitikz}
					\tikzstyle{every node}=[font=\normalsize]
					\draw [ line width=1.1pt ] (2.25,12.75) rectangle (6,11.5);
					\draw [ fill={rgb,255:red,0; green,0; blue,0} , line width=1.1pt ] (6,12.75) rectangle (6.25,11.5);
					\draw [ line width=1.1pt ] (6.25,12.75) rectangle (8,11.5);
					\draw [ fill={rgb,255:red,0; green,0; blue,0} , line width=1.1pt ] (8,12.75) rectangle (8.25,11.5);
					\node [font=\normalsize] at (4.25,12) {$Air, pressure P_1$};
					\draw [short] (2.25,13.5) -- (2.25,12.75);
					\draw [short] (6,13.5) -- (6,12.75);
					\draw [short] (2.25,11.5) -- (2.25,10.75);
					\draw [short] (8.25,11.5) -- (8.25,10.75);
					\draw [<->, >=Stealth] (2.25,11) -- (8.25,11);
					\draw [<->, >=Stealth] (2.25,13.25) -- (6,13.25);
					\node [font=\normalsize] at (4,13.5) {$L_1$};
					\node [font=\normalsize] at (4.75,10.75) {$L_2$};
					\draw [short] (6.25,12.75) .. controls (6.25,13.5) and (6,13.75) .. (6.5,13.75);
					\draw [short] (6.5,13.75) .. controls (7,14) and (7,14.25) .. (6.75,14.5);
					\draw [short] (6.25,12.75) -- (6.5,13);
					\draw [short] (6.25,12.75) -- (6,13);
					\draw [short] (8.25,12.75) -- (8.5,13);
					\draw [line width=1.1pt, short] (8.25,12.75) -- (9.25,12.75);
					\draw [line width=1.1pt, short] (8.25,11.5) -- (9.25,11.5);
					\draw [short] (8.25,13.5) .. controls (9,13.75) and (9,14) .. (8.75,14.25);
					\draw [short] (8.25,12.75) .. controls (8,13.25) and (7.75,13.5) .. (8.25,13.5);
					\draw [short] (8.25,12.75) -- (7.75,13);
					\node [font=\normalsize] at (6.25,15.25) {Initial position};
					\node [font=\normalsize] at (9.5,15) {Final position of};
					\node [font=\normalsize] at (6.25,14.75) {of the piston};
					\node [font=\normalsize] at (9.25,14.5) {the piston};
					\node [font=\normalsize] at (10.5,12.25) {Atmosphere};
					\node [font=\normalsize] at (10.5,11.75) {$pressure P_0$};
					\draw [ fill={rgb,255:red,255; green,190; blue,111} , line width=0.8pt ] (6.25,12) rectangle (7,11.5);
					\draw [ fill={rgb,255:red,255; green,190; blue,111} , line width=0.8pt ] (8.25,12) rectangle (9,11.5);
					\node [font=\normalsize] at (6.5,11.75) {$S_1$};
					\node [font=\normalsize] at (8.5,11.75) {$S_2$};
				\end{circuitikz}
				}%
		\end{figure}
		\begin{enumerate}
			\item 
		\end{enumerate}





\end{enumerate}
\end{document}
