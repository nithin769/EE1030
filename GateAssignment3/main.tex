\let\negmedspace\undefined
\let\negthickspace\undefined
\documentclass[journal]{IEEEtran}
\usepackage[a5paper, margin=10mm, onecolumn]{geometry}
%\usepackage{lmodern} % Ensure lmodern is loaded for pdflatex
\usepackage{tfrupee} % Include tfrupee package

\setlength{\headheight}{1cm} % Set the height of the header box
\setlength{\headsep}{0mm}     % Set the distance between the header box and the top of the text

\usepackage{gvv-book}
\usepackage{gvv}
\usepackage{cite}
\usepackage{amsmath,amssymb,amsfonts,amsthm}
\usepackage{algorithmic}
\usepackage{graphicx}
\usepackage{textcomp}
\usepackage{xcolor}
\usepackage{txfonts}
\usepackage{listings}
\usepackage{enumitem}
\usepackage{mathtools}
\usepackage{gensymb}
\usepackage{comment}
\usepackage[breaklinks=true]{hyperref}
\usepackage{tkz-euclide}
\usepackage{listings}
% \usepackage{gvv}
\def\inputGnumericTable{}
\usepackage[latin1]{inputenc}
\usepackage{color}
\usepackage{array}
\usepackage{longtable}
\usepackage{calc}
\usepackage{multirow}
\usepackage{hhline}
\usepackage{ifthen}
\usepackage{lscape}
\renewcommand{\thefigure}{\theenumi}
\renewcommand{\thetable}{\theenumi}
\setlength{\intextsep}{10pt} % Space between text and floats


\numberwithin{equation}{enumi}
\numberwithin{figure}{enumi}
\renewcommand{\thetable}{\theenumi}

% Marks the beginning of the document
\begin{document}
\bibliographystyle{IEEEtran}

\title{GateAssignment3}
\author{EE24BTECH11048-NITHIN.K} 
% \maketitle
% \newpage
% \bigskip
{\let\newpage\relax\maketitle}

\begin{enumerate}
\section{General Aptitude (GA) Questions(Compulsary)}
%TS f1
\item One of the parts $\brak{\text{A, B, C, D}}$ in the sentence given below contains an ERROR. Which one of the following is INCORRECT? \\
	\textbf{I requested that he should be given the driving test today instead of tomorrow.}
		\begin{enumerate}
			\item requested that
			\item should be given
			\item the driving test
			\item instead of tomorrow
		\end{enumerate}
%2
\item Which one of the following options is the closest in meaning to the word given below? \\
	\textbf{Latitude}
		\begin{enumerate}
			\item Eligibility
			\item Freedom
			\item Coercion
			\item Meticulousness
		\end{enumerate}
%3
\item Choose the most appropriate word from the options given below to complete the following sentence: \\
	\textbf{Given the seriousness of the situation that he had to face, his \underline{\hspace{1cm}} was impressive.}
		\begin{enumerate}
			\item beggary
			\item nomenclature
			\item jealousy
			\item nonchalance
		\end{enumerate}
%4
\item Choose the most appropriate alternative from the options given below to complete the following sentence: \\
	\textbf{If the tired soldier wanted to lie down, he \underline{\hspace{1cm}} the mattress out on the balcony.}
		\begin{enumerate}
			\item should take
			\item shall take
			\item should have taken
			\item will have taken
		\end{enumerate}
%5
\item If $\brak{1.001}^1259 = 3.52$ and $\brak{1.001}^2062 = 7.85$, then $\brak{1.001}^3321 =$
	\begin{enumerate}                                                                               
		\item 2.23
		\item 4.33
		\item 11.37
		\item 27.64
	\end{enumerate}
%6
\item A and B are friends. They decide to meet between 1 PM and 2 PM on a given day. There is a condition that whoever arrives first will not wait for the other for more than 15 minutes. The probability that they will meet on that day is
	\begin{enumerate}
		\item $\frac{1}{4}$
		\item $\frac{1}{16}$
		\item $\frac{7}{16}$
		\item $\frac{9}{16}$
	\end{enumerate}
%7
\item The data given in the following table summarizes the monthly budget of an average household.
	\begin{table}[h!]
		\centering
		\begin{center}
    \begin{tabular}{|c|c|c|c|} 
        \hline
            $\alpha \brak{\text{deg}}$ & $C_L$ & $C_D$ & $C_L$/$C_D$ \\
        \hline
            11 & 1.46 & 0.0865 & 16.9 \\
        \hline
            9 & 1.36 & 0.0675 & 20.1 \\
        \hline
            7 & 1.23 & 0.0535 & 22.9 \\
        \hline
            5 & 1.08 & 0.0440 & 24.5 \\
        \hline
	    3 & 0.90 & 0.0350 & 25.7 \\
	\hline
	    1 & 0.70 & 0.0275 & 25.4 \\
	\hline
            -1 & 0.49 & 0.0220 & 22.2 \\
	\hline
	    -3 & 0.25 & 0.0180 & 13.8 \\
	\hline
    \end{tabular}
\end{center}

	\end{table} \\
		The approximate percentage of the monthly budget \textbf{NOT} spent on savings is
		\begin{enumerate}
			\item 10\%
			\item 14\%
			\item 81\%
			\item 86\%
		\end{enumerate}
%8
\item There are eight bags of rice looking alike, seven of which have equal weight and one is slightly heavier. The weighing balance is of unlimited capacity. Using this balance, the minimum number of weighings required to identify the heavier bag is
	\begin{enumerate}
		\item 2
		\item 3
		\item 4
		\item 8
	\end{enumerate}
%9
\item Raju has 14 currency notes in his pocket consisting of only Rs. 20 notes and Rs. 10 notes. The total money value of the notes is Rs. 230. The number of Rs. 10 notes that Raju has is
	\begin{enumerate}
		\item 5
		\item 6
		\item 9
		\item 10
	\end{enumerate}
%10
\item \textbf{One of the legacies of the Roman legions was discipline. In the legions, military law prevailed and discipline was brutal. Discipline on the battlefield kept units obedient, intact and fighting, even when the odds and conditions were against them.} \\
Which one of the following statements best sums up the meaning of the above passage?
\begin{enumerate}
	\item Thorough regimentation was the main reason for the efficiency of the Roman legions even in adverse circumstances.
	\item The legions were treated inhumanly as if the men were animals.
	\item Discipline was the armies inheritance from their seniors.
	\item The harsh discipline to which the legions were subjected to led to the odds and conditions being against them.
\end{enumerate}
%11
\section{A : ENGINEERING MATHEMATICS (Compulsory)}
\item For the matrix $vec{M} = \myvec{1 & 4 & 5\\
0 & 2 & 6\\
0 & 0 & 3}$, consider the following statements: \\
P: 3 is an eigenvalue of M.   Q: $\myvec{4\\
1\\
0}$ is an eigenvector of M.   R: $\myvec{4\\
2\\
0}$ is an eigenvector of M. \\
Which of the above statements is \textbf{True}?
\begin{enumerate}
	\item P and Q, but not R
	\item Q and R, but not P
	\item P and R, but not Q
	\item P, Q, and R
\end{enumerate}
%12
\item Taylor series of $f\brak{x} = \frac{1}{1 - x}$ about $x = 0$ is given as $TS|_f = 1 + x + x^2 + x^3 + ...$ This series can be used to evaluate $f\brak{x}$ for
	\begin{enumerate}
		\item $\abs{x} \neq 1$
		\item $x < -1$
		\item $\abs{x} < 1$
		\item $-1 \leq x \leq 1$
	\end{enumerate}
%13
\item Let $f\brak{u, v} = u\ln{v}$ and $F\brak{x, y} = f\brak{u\brak{x, y}, v\brak{x, y}}$, where $u = \frac{x}{y}$ and $v = x - y$. Then $\frac{\partial F}{\partial x}$ is
	\begin{enumerate}
		\item $-\frac{x}{y^2}\ln{x - y} + \frac{x}{y\brak{x - y}}$
		\item $-\frac{x}{y^2}\ln{v} - \frac{u}{v}$
		\item $\frac{x}{y^2}\ln{v} - \frac{u}{V}$
		\item $\frac{x}{y^2}\ln{v} + \frac{x}{yv}$
	\end{enumerate}

\end{enumerate}
\end{document}
